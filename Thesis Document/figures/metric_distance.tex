\documentclass{article}
\usepackage[utf8]{inputenc}
\usepackage[usenames,dvipsnames,svgnames,table]{xcolor}
\usepackage[marginparwidth=28mm]{geometry}
\usepackage[pdftex]{graphicx}
\usepackage{tikz}

\usepackage{filecontents}
\usepackage[T1]{fontenc}
\usepackage[UKenglish]{babel}
\usepackage{newpxtext,newpxmath}
\usepackage[babel=true]{csquotes}
\usepackage[round]{natbib}
\usepackage[colorinlistoftodos]{todonotes}
\usepackage{comment}
\usetikzlibrary{arrows,automata}
\usepackage{ccaption}
\usepackage{url}
\usepackage{flafter}
\usepackage{pgfplots}
\begin{document}

\newcommand*{\xMin}{-6}%
\newcommand*{\xMax}{6}%
\newcommand*{\yMin}{-6}%
\newcommand*{\yMax}{6}%


\newcommand{\epuck}[3][0] % [angle]{x}{y} avec angle optionel
{
	\draw [very thick, fill=white] (#2,#3) circle [radius=0.5];
	\draw [very thick, rotate around={#1:(#2,#3)}] (#2-0.25,#3-0.433) -- (#2,#3+0.45) -- (#2+0.25,#3-0.433);
}

\newcommand{\epuckred}[3][0] % [angle]{x}{y} avec angle optionel
{
	\draw [very thick, fill=orange] (#2,#3) circle [radius=0.5];
	\draw [very thick, rotate around={#1:(#2,#3)}] (#2-0.25,#3-0.433) -- (#2,#3+0.45) -- (#2+0.25,#3-0.433);
}

\newcommand{\epuckblue}[3][0] % [angle]{x}{y} avec angle optionel
{
	\draw [very thick, fill=RoyalBlue] (#2,#3) circle [radius=0.5];
	\draw [very thick, rotate around={#1:(#2,#3)}] (#2-0.25,#3-0.433) -- (#2,#3+0.45) -- (#2+0.25,#3-0.433);
}

\newcommand{\human}[3][0] % [angle]{x}{y}
{
	\draw [very thick, fill=white, rotate around={#1:(#2,#3)}] (#2-1,#3+0.5) ellipse (0.25cm and 0.5cm);
	\draw [very thick, fill=white, rotate around={#1:(#2,#3)}] (#2+1,#3+0.5) ellipse (0.25cm and 0.5cm);
	\draw [very thick, fill=white, rotate around={#1:(#2,#3)}] (#2,#3) ellipse (1.5cm and 0.75cm);
	\draw [thick, fill=white, rotate around={#1:(#2,#3)}] (#2-0.05,#3+1) -- (#2,#3+1.1) -- (#2+0.05,#3+1);
	\draw [very thick, fill=white, rotate around={#1:(#2,#3)}] (#2,#3+0.5) circle [radius=0.5cm];
}
\begin{figure}\centering
\begin{tikzpicture}

	\foreach \i in {\xMin,...,\xMax} {
        \draw [very thin,gray] (\i,\yMin) -- (\i,\yMax)  node [below] at (\i,\yMin) {$\i$};
    }
    \foreach \i in {\yMin,...,\yMax} {
        \draw [very thin,gray] (\xMin,\i) -- (\xMax,\i) node [left] at (\xMin,\i) {$\i$};
    }

	\draw[red, fill] (-0.5,-2) rectangle (-2.35,2);
	\draw[LimeGreen, fill] (0.5,-2) rectangle (2.35,2);
	
	\draw[BurlyWood, fill] (-0.73,-1.77) rectangle (-2.12,1.77);
	\draw[BurlyWood, fill] (0.73,-1.77) rectangle (2.12,1.77);
	
%	\draw[dashed, very thick] (-2.35,5) -- (2.35,5);
%	\draw[dashed, very thick] (-2.35,-5) -- (2.35,-5);
%	\draw[dashed, very thick] (-5.35,2) -- (-5.35,-2);
%	\draw[dashed, very thick] (5.35,2) -- (5.35,-2);
%	\draw[dashed, very thick] (-5.35,2) arc (180:90:3);
%	\draw[dashed, very thick] (5.35,2) arc (0:90:3);
%	\draw[dashed, very thick] (-5.35,-2) arc (180:270:3);
%	\draw[dashed, very thick] (5.35,-2) arc (0:-90:3);
	
	\draw [dashed, very thick, orange] (2.35,5) to [out=0,in=90] (5.35,2) to [out=270,in=90] (5.35,-2) to [out=270, in=0] (2.35,-5) to [out=180, in=0] (-2.35,-5) to [out=180, in=270] (-5.35,-2) to [out=90, in=270] (-5.35,2) to [out=90, in=180] (-2.35,5) -- cycle;

	\draw[dashed, LightGray] (-2.37,-2.02) rectangle (2.37,2.02);
	
	\draw[<->, very thick] (0,2.02) -- (0,5) node [right] at (0,3.5) {\textsc{Target Distance}};
	\draw[<->, very thick] (-2.37,-2.02) -- (-4.47,-4.12);
	\draw[<->, very thick] (2.37,-2.02) -- (4.47,-4.12);
	
	\epuck{0}{-5}
	\epuck{-4.49}{4.14}
	\epuck{4.49}{4.14}
\end{tikzpicture}
\end{figure}

\end{document}