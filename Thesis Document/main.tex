\documentclass[a4paper, 12pt]{report}

\usepackage[utf8]{inputenc}
\usepackage{color}
\usepackage{geometry}
\usepackage[pdftex]{graphicx}
\usepackage{tikz}

\usepackage{filecontents}
\usepackage[T1]{fontenc}
\usepackage[UKenglish]{babel}
\usepackage{newpxtext} % DA AMAZING FONT


% Logos
\newcommand{\ulb}{\includegraphics[scale=1.1]{PageDeGarde_MFE/logo_ULB2.pdf}}
\newcommand{\polytech}{\includegraphics[scale=0.35]{PageDeGarde_MFE/logo_polytech_FR.pdf}}

% Polices
\definecolor{ULBblue}{rgb}{0,0.2196,0.5765}
\newcommand{\fontTitle}{\sffamily \Huge\selectfont \color{ULBblue}}
\newcommand{\fontSubtitle}{\sffamily \LARGE \selectfont \color{ULBblue}}
\newcommand{\fontText}{\sffamily \selectfont}
\newcommand{\fontColor}{\sffamily \selectfont \color{ULBblue}}

% Titre
\newcommand{\titleA}{\fontTitle{Human - Robots Swarms Interactions}} % Titre identique au titre remis au secrétariat
%\newcommand{\titleB}{\fontTitle{Deuxième ligne de titre du mémoire}} % (dans la langue de rédaction a priori)
% Sous-titre
\newcommand{\subtitle}{\fontSubtitle{Ligne du sous-titre du mémoire}}
% Titre du diplôme
\newcommand{\diplomaA}{\fontText{Mémoire présenté en vue de l’obtention du diplôme}} % A laisser en Français
\newcommand{\diplomaB}{\fontText{d'Ingénieur Civil en Informatique à finalité Intelligence Computationnelle}}

% Etudiant
\newcommand{\student}{\textbf{\sffamily \large Anthony Debruyn}}

% Supervision
\newcommand{\promAa}{\fontColor{Directeur}}
\newcommand{\promAb}{\fontText{Professeur [Prénom Nom]}}
\newcommand{\promBa}{\fontColor{Co-Promoteur}}
\newcommand{\promBb}{\fontText{Professeur [Prénom Nom]}}
\newcommand{\promCa}{\fontColor{Superviseur}}
\newcommand{\promCb}{\fontText{[Prénom Nom]}}
\newcommand{\deptA}{\fontColor{Service}}
\newcommand{\deptB}{\fontText{[Nom du service]}}

% Année académique
\newcommand{\yearA}{\fontColor{Année académique}}
\newcommand{\yearB}{\fontText{2014 - 2015}}

\begin{document}

	\thispagestyle{empty}
	\newgeometry{top=2.5cm, bottom=1.5cm, left=2.5cm, right=1cm}
	\setlength{\unitlength}{1mm}
	\noindent\begin{picture}(175,257)
	
		\put(0,245){\polytech}
		\put(153,139.5){\ulb}
		
		\put(8,155){\makebox(150,10)[l]{\titleA}}
%		\put(8,145){\makebox(150,10)[l]{\titleB}}
		\put(8,135){\makebox(150,10)[l]{\subtitle}}
		
		\put(0,75){
		\begin{tikzpicture}[scale=0.1]
		\fill [fill=ULBblue](0,0) rectangle (0.8,90);
		\fill [fill=ULBblue](0,57) rectangle (152,57.8);
		\end{tikzpicture}}
		
		\put(8,120){\makebox(150,5)[l]{\diplomaA}}
		\put(8,115){\makebox(150,5)[l]{\diplomaB}}
		
		\put(8,75){\makebox(150,10)[l]{\selectfont \student}}
		
		\put(8,44){\makebox(80,5)[l]{\promAa}}
		\put(8,39){\makebox(80,5)[l]{\promAb}}
		\put(8,31){\makebox(80,5)[l]{\promBa}} % Commenter la ligne si pas nécessaire
		\put(8,26){\makebox(80,5)[l]{\promBb}} % Commenter la ligne si pas nécessaire
		\put(8,18){\makebox(80,5)[l]{\promCa}} % Commenter la ligne si pas nécessaire
		\put(8,13){\makebox(80,5)[l]{\promCb}} % Commenter la ligne si pas nécessaire
		\put(8,5){\makebox(80,5)[l]{\deptA}}
		\put(8,0){\makebox(80,5)[l]{\deptB}}
		
		\put(145,5){\makebox(30,5)[r]{\yearA}}
		\put(145,0){\makebox(30,5)[r]{\yearB}}
	
	\end{picture}
	\restoregeometry
	
% Template conçu par Benjamin Vanhemelryck et revu par François Bronchart - Mai 2013
	
%%%%%%%%%%%%%%%%%%%%%%%%%%%%%%%%%%%%%%%%%%%%%%%%%%%%%%%
% TABLE OF CONTENT
%%%%%%%%%%%%%%%%%%%%%%%%%%%%%%%%%%%%%%%%%%%%%%%%%%%%%%%

\tableofcontents
	
%%%%%%%%%%%%%%%%%%%%%%%%%%%%%%%%%%%%%%%%%%%%%%%%%%%%%%%
% INTRODUCTION
%%%%%%%%%%%%%%%%%%%%%%%%%%%%%%%%%%%%%%%%%%%%%%%%%%%%%%%

\chapter{Introduction}

[To write at the end. When I have a global view.]

\chapter{The Problem}
	\section{Robots - Human Feedback}
	
[Currently, only little research on feedback between human and robots swarms. That research is focused on interaction with an additional layer... Mostly unidirectional communication (human to robots). Need new types of interactions for new applications. Robots to human (guide). Write about article saying self-organised feedback is better.]

	\section{Protection and Exploration}

\chapter{State of the Art}
	\section{Introduction}
	\section{Swarm Intelligence/Robotics}
	\section{Design Methods}
	\section{Feedbacks}

\chapter{Solution}

	\section{Introduction}
	\section{Swarm Robotics}
	\section{The Shape}
	\section{Virtual Physics}
	\section{Potentials}
		\subsection{Human Potential}
		\subsection{Gravity Potential}
		\subsection{Agent Repulsion Potential}
		\subsection{Default Potential}
	\section{Range and Bearing Sensor}
		\subsection{Tests}
		\subsection{Calibration}
		\subsection{Conclusions}

	\section{Omnidirectional Camera Sensor}
		\subsection{Tests}

	\section{The Hardware}
		\subsection{Objective}
		\subsection{Choices}
		\subsection{Blueprints}
		\subsection{Build Process}

\chapter{Assessments}
	\section{Metrics}
	\section{More robots ?}
	\section{Tests Scenarios}
	\section{Human Travelling Speed ?}

\chapter{Future Works}
	\section{Other Robots}
	\section{Human Guidance}
		\subsection{Zero Visibility Areas or Blind People}
		\subsection{Human Motion Synchronisation}

\chapter{Acknowledgements}

[Never gonna give you up\\
Never gonna let you down\\
Never gonna run around and desert you\\
Never gonna make you cry\\
Never gonna say goodbye\\
Never gonna tell a lie and hurt you]

\newpage

%%%%%%%%%%%%%%%%%%%%%%%%%%%%%%%%%%%%%%%%%%%%%%%%%%%%%%%
% BIBLIOGRAPHY
%%%%%%%%%%%%%%%%%%%%%%%%%%%%%%%%%%%%%%%%%%%%%%%%%%%%%%%

\begin{filecontents}{thesis.bib}
%http://en.wikibooks.org/wiki/LaTeX/Bibliography_Management


@article{Brambilla2013a,
    author    = {Manuele Brambilla and Eliseo Ferrante and Mauro Birattari and Marco Dorigo},
    title     = {Swarm robotics: a review from the swarm engineering perspective},
    journal   = {Swarm Intelligence},
    %volume   = {},
    %number   = {},
    %pages    = {},
    year      = {2013},
    %month    = {},
    %note     = {},
}

@inproceedings{podevijn2012self,
  title={Self-organised feedback in human swarm interaction},
  author={Podevijn, Ga{\"e}tan and O’Grady, Rehan and Dorigo, Marco},
  booktitle={Proceedings of the workshop on robot feedback in human-robot interaction: how to make a robot readable for a human interaction partner (Ro-Man 2012)},
  year={2012}
}

@incollection{podevijn2014gesturing,
  title={Gesturing at subswarms: Towards direct human control of robot swarms},
  author={Podevijn, Ga{\"e}tan and O’Grady, Rehan and Nashed, Youssef SG and Dorigo, Marco},
  booktitle={Towards Autonomous Robotic Systems},
  pages={390--403},
  year={2014},
  publisher={Springer}
}

\end{filecontents}

\nocite{*}
\bibliographystyle{alpha}
\bibliography{thesis}

\end{document}
